\documentclass[10pt,xcolor={dvipsnames}]{beamer}

%\includeonlyframes{current}

\usetheme{blei} % at ~/texmf/tex/latex/local/
\addbibresource{library.bib}

% https://tex.stackexchange.com/a/1475/193011
% PGF doesn't use TikZ, but the tikzplotlib backend does
%\usetikzlibrary{external}
%\tikzexternalize[prefix=tikz/]

%%%%%%%%%%
% \notes %
%%%%%%%%%%
\usepackage{pgfpages}
%\setbeameroption{hide notes} % Only slides
%\setbeameroption{show only notes} % Only notes
%\setbeameroption{show notes on second screen=right} % Both
\setbeamertemplate{note page}{\pagecolor{yellow!5}\vfill\insertnote\vfill}

% https://latex.org/forum/viewtopic.php?t=11357
\usepackage{xparse}
\NewDocumentCommand\notes{>{\SplitList{;}}m}
{
	\note{%
    \begin{itemize}
      \ProcessList{#1}{ \insertitem }
    \end{itemize}
    }
}
\newcommand\insertitem[1]{\item #1}


%%%%%%%%%%%%
% Packages %
%%%%%%%%%%%%
\usepackage{amsmath}
\usepackage{physics}
\usepackage{pgf}
\DeclareUnicodeCharacter{2212}{-} % Needed for matplotlib's pgf output
\usepackage{pythonhighlight}
\usepackage{subfig}
\usepackage{marvosym}
\usepackage[normalem]{ulem}
\usepackage{bbding}
\usepackage{ragged2e}

%\usepackage{bm} % Clashes with main font, so use \boldsymbol
\input{../overleaf/definitions.tex}
\newcommand{\bm}{\boldsymbol}

\newcommand{\pgffigure}[2]{%
\scalebox{#2}{\input{img/#1}}
}

\newcommand{\citelink}[2]{\href{#1}{\small\color{gray}[#2]}}

\definecolor{mplblue}{HTML}{1f77b4}
\definecolor{mplorange}{HTML}{ff7f0e}

% https://tex.stackexchange.com/a/326380/193011
\usepackage{xcolor}
% Syntax: \colorboxed[<color model>]{<color specification>}{<math formula>}
\newcommand*{\colorboxed}{}
\def\colorboxed#1#{%
  \colorboxedAux{#1}%
}
\newcommand*{\colorboxedAux}[3]{%
  % #1: optional argument for color model
  % #2: color specification
  % #3: formula
  \begingroup
    \colorlet{cb@saved}{.}%
    \color#1{#2}%
    \boxed{%
      \color{cb@saved}%
      #3%
    }%
  \endgroup
}

%%%%%%%%%
% Title %
%%%%%%%%%

\title{A weakly informative prior\\for resonance frequencies}
\date{MaxEnt 2021}
\author{Marnix Van Soom \& Bart de Boer}
%\institute{MaxEnt 2021}
\titlegraphic{\vspace{.75cm}\includegraphics[height=1.8cm]{logo/VUB_RGB.png}}

\begin{document}

\maketitle
\notes{%
In my own work on high precision measurement of resonances of the vocal tract I came across the problem of choosing a prior for frequencies;
It turned out to be a subtle problem;
And I want to share a Maximum Entropy solution we came up with with you in the hope that it will be applicable in other areas of science
}

%\section{Motivation}

\begin{frame}{Resonance frequencies in the natural sciences}
\pause
\begin{columns}
\begin{column}{.25\textwidth}
\href{https://en.wikipedia.org/wiki/Nuclear_magnetic_resonance_spectroscopy}{\includegraphics[width=\textwidth]{img/nmr.jpg}}
\cite{Wilson2014a}
\pause
\end{column}
\begin{column}{.25\textwidth}
\href{https://en.wikipedia.org/wiki/LIGO}{%
\includegraphics[width=\textwidth]{img/ligo1.jpg}
\includegraphics[width=\textwidth]{img/ligo2.jpg}
}
\cite{Littenberg2015}
\pause
\end{column}
\begin{column}{.25\textwidth}
\href{https://acoustics.org/pressroom/httpdocs/158th/pestka.htm}{ \includegraphics[width=\textwidth]{img/rus.jpg}}
\cite{Xu2019}
\end{column}
\end{columns}

\notes{%
Three examples of RF in science together with recent papers giving Bayesian approach to measuring them;
Nuclear magnetic resonance (NMR): Resonances give information about the structure of organic molecules in solution. Used in MRI scans;
LIGO/Virgo: Gravitational wave data from ground-based detectors is dominated by instrument noise. Sharp resonances occur on top of broadband features which must be taken into account when instrument noise is modeled;
Resonant Ultrasound Spectroscopy (RUS): Measure fundamental material properties involving elasticity. Elastic tensor of material.
}
\end{frame}

\begin{frame}{Measurement model}
\pause
\greybox{$p(D,\bx) \onslide<5->{= \likelihood(\bx) \ \pi(\bx)}$}
\pause
\vspace{1cm}
\begin{overprint}
\onslide<3-4>
\begin{columns}
\begin{column}{.5\textwidth}
\greybox{%
\begin{align*}
	\onslide<3->{D &= \text{data}} \\
	\onslide<4->{\bx &= \{x_1\cdots x_K\}} \\
	\onslide<4->{&= \text{$K$ resonance frequencies}}
\end{align*}
}
\end{column}
\begin{column}{.5\textwidth}
\onslide<3->{\pgffigure{model-data.pgf}{.5}}
\onslide<4->{\pgffigure{model-spectrum.pgf}{.5}}
\end{column}
\end{columns}

\onslide<6->
\begin{itemize}
\widesep
\item<6-> Assume $\likelihood(\bx)$ is given
\item<7-> \alert{\textsc{choose $\pi(\bx)$ subject to limited prior information}}
\vspace{.25cm}
\begin{itemize}
	\item<8-> Don't know prior estimates $\hat{\bx}$
	\item<9-> Don't know $K$
$$ Z(K) = \int \dd[K]{\bx} \likelihood(\bx) \pi(\bx) $$
\end{itemize}
\end{itemize}
\end{overprint}

\notes{%
Model of the measurement process is defined by the joint of the data and the resonance frequencies;
Waveform of a male saying /ae/ in "that";
In our example the RF $\bx$ parametrize the frequency responce of the vocal tract;
Our task is to choose $\pi(\bx)$ from limited prior information: a common situation;
We look at three choices
}

\end{frame}

\begin{frame}{First candidate: $\pi_1(\bx|\cdot)$}
\pause
\vspace{.5cm}
\greybox{$\pi_1(\bx|a,b) = \prod_{k=1}^K h(x_k|a,b)$}
\pause
\vspace{1cm}
\begin{overprint}
\onslide<2-7>
\begin{itemize}[<+->]
	\widesep
	\item Most common choice in literature
	\item $K$ iid $x_k \in [a,b]$ with \emph{global} bounds
	\item \alert{Label switching problem} \Lightning
	\begin{itemize}
	\item Exchange symmetry
	\item Frustrates calculating $Z(K)$ \cite{Celeux2018}
	\end{itemize}
\end{itemize}
\onslide<8>
\hspace{-.35cm}
\pgffigure{label-switching-talk-1.pgf}{.58}
\onslide<9>
\hspace{-.35cm}
\pgffigure{label-switching-talk-2.pgf}{.58}
\onslide<10>
\hspace{-.35cm}
\pgffigure{label-switching-talk-3.pgf}{.58}
\end{overprint}

\notes{%
Global bounds are typically lower bound and the Nyquist frequency;
If $\likelihood$ has exchange symmetry, so will posterior;
This makes reliable calculation of $Z(K)$ hard: problem well-known from mixture modelling;
Example picture shows posterior degeneration for $K=3$ with Jeffreys priors for $h(x|a,b)$
}

\end{frame}

\begin{frame}{Second candidate: $\pi_2(\bx|\cdot)$}
\pause
\greybox{$\pi_2(\bx|\ba,\bb) = \prod_{k=1}^K h(x_k|a_k,b_k)$}
\pause
\vspace{1cm}
\begin{itemize}[<+->]
	\widesep
\item $K$ independent $x_k \in [a_k,b_k]$ with \emph{local} bounds $(\ba,\bb)$
	\item \alert{Multiplet problem} \Lightning
\begin{itemize}[<+->]
\item Need overlap for close frequencies
\item But this brings back the label switching problem
\end{itemize}
\end{itemize}

\notes{%
Break the exchange symmetry with local, specialized bounds;
This requires extra knowledge or some heuristic;
But this does not allow detection of multiplet of arbitrary order;
Nonetheless, is useful if knowledge is available and high-order multiplets are not expected
}
\end{frame}

\begin{frame}{Third candidate: $\pi_3(\bx|\cdot)$}
\vspace{.2cm}

\onslide<2->
\greybox{$\pi_3(\bx|\blambda) = \prod_{k=1}^K \Pareto(x_k | x_{k-1},\lambda_k) \qquad \onslide<5->{\bx \in \orderedregion_K(x_0)}$}

\vspace{.5cm}

\begin{overprint}

\onslide<3->
\begin{itemize}
\widesep
\item<3-> \alert{Maximum entropy distribution}

% https://tex.stackexchange.com/a/306233/193011
\item<4->
\begin{minipage}[t]{.6\textwidth}
Chain of coupled \emph{Pareto distributions}
$$ \Pareto(x|x_*,\lambda) = \frac{\lambda}{x} \qty(\frac{x_*}{x})^\lambda  \qquad (x_* \leq x) $$
\end{minipage}
\begin{minipage}[t]{.3\textwidth}
\vspace{-.3cm}
\begin{figure}
	\pgffigure{pareto-pdf.pgf}{.6}
\end{figure}
\end{minipage}
\vspace{-.5cm}
\item<5->
\begin{minipage}[t]{.6\textwidth}
Supported by the \emph{ordered region}
$$\orderedregion_K(x_0) = \{\bx | x_0 \leq x_1 \leq x_2 \leq \cdots \leq x_K\}$$
\end{minipage}
\begin{minipage}[t]{.3\textwidth}
\vspace{-.3cm}
\begin{figure}
\pgffigure{ordered-region-2d.pgf}{.6}
\end{figure}
\end{minipage}

\end{itemize}

\onslide<6->
\hspace{-.35cm}
\pgffigure{joint-talk-region.pgf}{.58}
\begin{center}
Support $\orderedregion_3(x_0) = \{\bx | x_0 \leq x_1 \leq x_2 \leq x_3\}$
\end{center}

\onslide<7->
\hspace{-.35cm}
\pgffigure{joint-talk.pgf}{.58}
\begin{center}
Pairwise prior distribution $\textcolor{mplorange}{\pi_3(\bx|\blambda)}$
\end{center}

\onslide<8->
\hspace{-.35cm}
\pgffigure{label-switching-talk-4.pgf}{.58}
\begin{center}
Pairwise posterior distribution $\textcolor{mplorange}{P_3(\bx|\blambda)} \propto \likelihood(\bx) \textcolor{mplorange}{\pi_3(\bx|\blambda)}$
\end{center}

\onslide<9->
\hspace{-.35cm}
\pgffigure{label-switching-talk-3.pgf}{.58}
\begin{center}
Pairwise posterior distribution $\textcolor{mplblue}{P_1(\bx|a,b)} \propto \likelihood(\bx) \textcolor{mplblue}{\pi_1(\bx|a,b)}$
\end{center}

\onslide<10->
\hspace{-.35cm}
\pgffigure{label-switching-talk-5.pgf}{.58}
\begin{center}
$\textcolor{mplblue}{P_1(\bx|a,b)}$ vs. $\textcolor{mplorange}{P_3(\bx|\blambda)}$
\end{center}

\onslide<11->{%
\begin{columns}
\begin{column}{.5\textwidth}
\centering
\sout{Label switching problem} \textcolor{mplorange}{\CheckmarkBold}
\end{column}
\begin{column}{.5\textwidth}
\centering
\sout{Multiplet problem} \textcolor{mplorange}{\CheckmarkBold}
\end{column}
\end{columns}
}

\end{overprint}

\notes{%
Pareto distribution a.k.a. power low;
Maximum entropy distribution: in exponential family;
$\blambda$ are Lagrange multipliers, but we'll express them in terms of expected values of $\bx$;
Compression of prior to posterior in a well-posed problem;
Compare to performance of $\pi_1$;
Label switching problem of $\pi_1$, multiplet problem of $\pi_2$
}

\end{frame}

\begin{frame}{Deriving $\pi_3(\bx|\cdot)$}
\pause

\begin{overprint}
\onslide<+->

\greybox{%
Ansatz: Jeffreys prior $m(x_k) \propto 1/x_k$
}

\vspace{.8cm}

\begin{itemize}[<+->]


\item
\begin{minipage}[t]{.6\textwidth}
$
    m(\bx) = \left\{\begin{array}{ll}
        \prod_{k=1}^K \frac{1}{x_k} & \bx \in \orderedregion_K(x_0)\\
        0 & \text{else}
        \end{array}
$
\end{minipage}
\begin{minipage}[t]{.3\textwidth}
\vspace{-1cm}
\centering $\orderedregion_3(x_0)$
\begin{figure}
	\includegraphics[height=2cm]{img/oc-x.pdf}
\end{figure}
\end{minipage}

\vspace{-.5cm}
\item
$ \bx \rightarrow \bu\colon \quad \left\{\begin{aligned}
        u_1 &= \log(x_1/x_0) \\
        u_2 &= \log(x_2/x_1) \\
        &\cdots \\
        u_K &= \log(x_K/x_{K-1})
        \end{aligned} $

\vspace{1cm}
\item
\begin{minipage}[t]{.6\textwidth}
$
    m(\bu) = m(\bx(\bu)) \abs{\dv{\bx}{\bu}} = \left\{\begin{array}{ll}
        1 & \bu \geq 0 \\
        0 & \text{else}
        \end{array}
$
\end{minipage}
\begin{minipage}[t]{.3\textwidth}
\vspace{-1.5cm}
\centering $\orderedregion_3(x_0)$ in $\bu$ space
\begin{figure}
	\includegraphics[height=2cm]{img/oc-u.pdf}
\end{figure}
\end{minipage}

\end{itemize}

\onslide<+->

\greybox{%
Normalize $
    m(\bu) = \left\{\begin{array}{ll}
        1 & \bu \geq 0 \\
        0 & \text{else}
        \end{array}
$ by constraining first moments $\expval{\bu} := \bbaru$
}

\begin{itemize}[<+->]
\widesep

\item
Minimize
$$D_\text{KL}(\pi_3 | m) = \int \dd[K]{\bu} \pi_3(\bu) \log{\frac{\pi_3(\bu)}{m(\bu)}}$$
subject to
$$
\expval{\bu} \equiv \int \dd[K]{\bu} \bu \pi_3(\bu) := \bbaru
$$

\item
Solution:
$$ \pi_3(\bu|\blambda) = \prod_{k=1}^K \Exp(u_k|\lambda_k) \qquad (\bu \geq 0) $$
where the rates $\lambda_k = 1/\baru{k}$

\item Equivalent to Jaynes' \emph{principle of maximum entropy} with $m(\bu)$ serving as the invariant measure \cite{Jaynes1968}
\end{itemize}

\onslide<+->

\greybox{%
Transform $\pi_3(\bu|\blambda)$ to $\bx$ space and re-express $\blambda$
}

\vspace{.5cm}

\begin{itemize}[<+->]
\widesep

\item
$ \bu \rightarrow \bx\colon \quad \left\{\begin{aligned}
        x_1 &= x_0 \exp{u_1} \\
        x_2 &= x_0 \exp{u_1 + u_2} \\
        &\cdots \\
        x_K &= x_0 \exp{u_1 + u_2 + \cdots + u_K}
        \end{aligned} $

\item
$
\pi_3(\bx|\blambda) = \pi_3(\bu(\bx)|\blambda) \abs{\dv{\bu}{\bx}}
= \prod_{k=1}^K \Pareto(x_k | x_{k-1},\lambda_k)
$

\item How to set \alert{hyperparameters} $\lambda_k^{-1} = \baru{k} = \overline{\log{x_k/x_{k-1}}}$?
\begin{itemize}
\widesep
\item Use identity:
$
	\expval{x_k} \equiv \int \dd[K]{\bx} x_k \pi_3(\bx|\blambda) = \frac{\lambda_k}{\lambda_k-1} \expval{x_{k-1}}
$
\item Thus:
$$ \colorboxed{gray}{\lambda_k = \frac{\barx{k}}{\barx{k}-\barx{k-1}}} $$
\end{itemize}

\end{itemize}

\end{overprint}

\notes{%
Thus we can express $\blambda$ in something more easy to set: the expected values of the frequencies
}

\end{frame}

\begin{frame}{Further properties of $\pi_3(\bx|\cdot)$}
\pause

\begin{itemize}
\widesep
\item<+-> \alert{Convenient parametrization}:~ $\colorboxed{gray}{\pi_3(\bx|\bbarxnull) \qq{where} \bbarxnull \equiv (x_0,\barx{1},\barx{2},\cdots,\barx{K})}$

\item<+-> \alert{Inference insensitive} to values of the hyperparameters $\bbarxnull$
\begin{itemize}
\item Maximum entropy: weak inductive bias
\item Heavy tails
\item Terrible sample statistics $\frac{1}{n}\sum_i x_k^{(i)} \rightarrow \barx{k}$
\end{itemize}

\item<+-> \alert{Sampling} trivial: $\bu \rightarrow \bx$

\item<+-> \alert{``Consistent''}
\begin{itemize}
\item Marginalizing out higher frequencies $\equiv$ having set up $\pi_3(\bx|\bbarxnull)$ without knowledge of those frequencies
\end{itemize}

\item<+-> \alert{Scale invariant}: $\pi_3(c\bx|\bbarxnull) = f(c) \pi_3(\bx|\bbarxnull)$ \cite{Newman2005}

\end{itemize}


\end{frame}

%\section{A weakly informative prior}

%\section{Measuring resonance frequencies of the vocal tract}

% Show that diff in hypotheses is of O(K!) thus robust evaluation needed

%\section{Conclusion}

\begin{frame}[label=current]{Application}

\greybox{%
Compare the $\pi_i(\bx|\cdot)$ candidates on a simple inference task.
}
\vspace{.5cm}
\onslide<+->

\begin{overprint}

\onslide<+->

\begin{columns}
\begin{column}{.5\textwidth}
\begin{itemize}
\widesep
\item Measure resonance frequencies of the human vocal tract
\item Five representative vowel sounds taken from the CMU ARCTIC database \cite{Kominek2004}
\item $D \in \{\shore, \that, \you, \little, \until\}$
\item Compare $Z_i(K)$ and $H_i(K)$ for each $D$ and $i \in \{1,2,3\}$
\end{itemize}
\end{column}
\begin{column}{.5\textwidth}
\centering ``$\that$'' \\
\vspace{.1cm}
\pgffigure{model-data.pgf}{.5}
\pgffigure{model-spectrum.pgf}{.5}
\end{column}
\end{columns}

\onslide<+->

\begin{figure}[t]
\pgffigure{priors.pgf}{.62}
\caption{%
\footnotesize
\justifying
\textcolor{gray}{%
Comparison of $\pi_1$, $\pi_2$ and $\pi_3$ in terms of the marginal priors $\pi_i(x_k|\cdot)$ for the case $K :=3 $.
%The actual values of the hyperparameters used for this plot are given in Table~\ref{hyperparameters}.
The marginal $\pi_i(x_k|\cdot)$ is obtained by integrating out the two other frequencies; for example, $\pi_i(x_1|\cdot) = \iint \dd{x_2} \dd{x_3} \pi_i(\bx|\cdot)$.
The pdfs are shown on a common log scale and are scaled by the appropriate Jacobian determinant $\abs{\dv*{x_k}{\log{x_k}}} = x_k$.
\label{priors}
}
}
\end{figure}

\begin{center}
\alert{\textcolor{red}{$\pi_1(\bx|\cdot)$ is excluded for $K \geq 4$}}
\end{center}

\onslide<+->
\begin{figure}[t]      
\centering
\subfloat[]{\pgffigure{bars.pgf}{.62}}
\qquad
\subfloat[]{\pgffigure{arrow.pgf}{.62}}
\caption{%
\footnotesize
\justifying
\textcolor{gray}{%
{\bf (a)}
Model selection in Experiment I (top row) and Experiment II (bottom row).
{\bf (b)}
In Experiment I, $\pi_2$ and $\pi_3$ are compared in terms of evidence [$\log{Z_i(K)}$] and uninformativeness [$H_i(K)$] for each $(D,K)$.
The arrows point from $\pi_2$ to $\pi_3$ and are color-coded by the value of $K$.
For small values of $K$, the arrow lengths are too small to be visible on this scale.
}
}
\end{figure}

\onslide<+->
\begin{figure}[t]
\pgffigure{spectrum.pgf}{.62}
\caption{%
\footnotesize
\justifying
\textcolor{gray}{%
The VTR problem for the case $(D := \until, K := 10)$.
Left panel:
The data $D$, i.e., the quasi-periodic steady-state part consisting of 3 highly correlated pitch periods.
Right panel:
Inferred VTR frequency estimates $\{\hat{x}_k\}_{k=1}^K$ for $K := 10$ at 3 sigma.
They describe the power spectral density of the vocal tract transfer function $|T(x)|^2$, represented here by 25 posterior samples and compared to the Fast Fourier Transform (FFT) of $D$.
All $\hat{x}_k$ are well resolved and most have error bars too small to be seen on this scale.
%$\log{Z} = \numb{220.71(28)}$.
%$H = \numb{39}~\text{nats}$.
%SNR (dB) = 18.29(23).
\label{spectrum}
}
}
\end{figure}

\onslide<+->

\begin{center}
\alert{Conclusions:}\\
\end{center}

\begin{enumerate}
\widesep
\item $\pi_1$ can't be used for $K \geq 4$
\item $\pi_2$ dominated by $\pi_3$ in terms of \alert{evidence} $Z_i(K)$
\item $\pi_2$ about as uninformative as $\pi_3$ in terms of the \alert{information} $H_i(K)$
\item $\pi_3$ can push $K$ further
\end{enumerate}

\end{overprint}
\end{frame}

\begin{frame}{Summary}
\pause
%quantify the predictive quality of our assumption = evidence
%WHY BOTHER? WE HAVE SO MUCH DATA.

\begin{itemize}
\widesep

\item<2-> The prior facilitates model selection problems in which the number $K$ of resonance frequencies is unknown by enabling the use of more robust evidence-based methods, even in the presence of multiplets of arbitrary order.
\begin{enumerate}
\item Solves label switching problem
\item Solves multiplet problem
\end{enumerate}

\item<3-> The prior
\begin{enumerate}
\item is in the exponential family,
\item encodes a weakly inductive bias,
\item provides a reasonable density everywhere,
\item is easily parametrizable,
\item is easy to sample from.
\end{enumerate}
That's enough! Meant to be overwhelmed.

\item<4-> The prior is valid for any collection of scale variables which are intrinsically ordered.
\begin{itemize}
\item Does it apply to modeling spectra directly?
\end{itemize}

\end{itemize}

\end{frame}

\insertreferences

%\appendix

\end{document}
